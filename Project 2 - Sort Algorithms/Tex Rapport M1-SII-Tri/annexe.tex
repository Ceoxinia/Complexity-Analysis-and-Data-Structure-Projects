\chapter{Annexe}
\section{Repartition des taches}
\par
\small
\resizebox{17cm}{!}{
\begin{tabular}{| c | c | }
    \hline
    Member &  Tache   \\
    \hline
    BEBNBACHIR Mohamed Amir &  Implimentation des algorithmes et l'etude expiremental selection et a bulle  \\
    \hline
    BOUCHOUL Bouchra &  L'étude théorique et expérimentale et implémentation de Tri TAS et insertion  \\
    \hline
    KHEMISSI Maroua & L'étude théorique du du tri par selection ,fusion et bulle ,implimentation et etude experimentale de tri fusion  \\
    \hline
    MEDJKOUNE Roumaissa & L'étude théorique et expérimentale et implémentation de Tri Rapide  \\
    \hline
  
\end{tabular}}
\\ \\
\newpage
\section{files.c}
Le programme c qui genere les fichiers des donnees 
\begin{minted}[
breaklines=true,
frame=lines,
linenos
]{c}
#include <stdio.h>
#include <stdlib.h>
#define Length 50000000
int main()
{
    long i;
    FILE *f;
    f=fopen("tablerand.txt","w");

    /*Initialization of the array  */
    for(i=0;i<Length;i++){
       fprintf(f, "%d ", rand());
       //fprintf(f, "%d ", i); for ordered
       //fprintf(f, "%d ", Length-i); for inverse
    }
    fclose(f);

}
\end{minted}
\section{main.c}
\begin{minted}[
breaklines=true,
frame=lines,
linenos
]{c}
#include <stdio.h>
#include <stdlib.h>
#include <time.h>
#define Length 50000000

typedef long *array;
array A[Length];
array Tmp[Length];

/*********** Procedure Merge ************/
void merge(array a, long first,long last){
   long i = first;
   long med = (first+last)/2;
   long j = med+1;
   long k = 0;
   while (i<=med && j<=last){
   	if (a[i] < a[j]){
	   Tmp[k] = a[i];
	   i++;
	   k++;
	   }
	else{
	   Tmp[k] = a[j];
	   k++;
	   j++;}
    }
   while (i<=med) {Tmp[k] = a[i]; k++;i++;}
   while (j<=last) {Tmp[k] = a[j]; k++; j++;}

   for (j=0; j<k; j++)
	a[first+j] = Tmp[j];
}
/*********** Procedure merge sort ************/
void mergeSort(array a, long first, long last)
{
   long middle,i,j;
   if (first>=last) return;
   middle = (first+last)/2;
   mergeSort(a,first,middle);
   mergeSort(a,middle+1,last);
   merge(a,first,last);
}
/********* bubblesort*******/
void bubsort(int* T,int n){
    int  temp;
    int b;
    for (int i =1;i<n;i++){
        b=1;
        for (int j=0;j<n-i;j++){
            if (T[j]>T[j+1]){
                b=0;
            temp=T[j];
            T[j]=T[j+1];
            T[j+1]=temp;}
        }
        if(b)break;
        
    }
}
/********** selection ********/
void selection(int* T,int n){
    int  min,temp;
    for (int i =0;i<n;i++){
        min =i;
        for (int j=i+1;j<n;j++){
            if (T[j]<T[min]){
                min=j;
            }
        }
        temp=T[min];
        T[min]=T[i];
        T[i]=temp;
        
    }
}

/*********** tri insertion ********/
void insertion(long *T, long n) {
  long j, temp;
  for (long i = 1; i < n; i++) {
    temp=T[i];
    j=i-1;
    while ((T[j]>temp) && (j>=0)) {
            T[j+1]=T[j];
            j=j-1;
            }
      T[j+1]=temp;
  }
}
/************ tri tas **********/
void swap(long* a, long* b)
{
 
    long temp = *a;
 
    *a = *b;
 
    *b = temp;
}
 
// To heapify a subtree rooted with node i
// which is an index in arr[].
// n is size of heap
void heapify(long arr[], long N, long i)
{
    // Find largest among root, left child and right child
 
    // Initialize largest as root
    long largest = i;
 
    // left = 2*i + 1
    long left = 2 * i + 1;
 
    // right = 2*i + 2
    long right = 2 * i + 2;
 
    // If left child is larger than root
    if (left < N && arr[left] > arr[largest])
 
        largest = left;
 
    // If right child is larger than largest
    // so far
    if (right < N && arr[right] > arr[largest])
 
        largest = right;
 
    // Swap and continue heapifying if root is not largest
    // If largest is not root
    if (largest != i) {
 
        swap(&arr[i], &arr[largest]);
 
        // Recursively heapify the affected
        // sub-tree
        heapify(arr, N, largest);
    }
}
void heapsort(long *T, long n) {
  long temp;
  for (long k = n / 2-1; k >= 0; k--) {
    heapify(T, n, k);
  }
  for (long wn = n - 1; wn >= 0; wn--) {
    temp = T[0];
    T[0] = T[wn];
    T[wn] = temp;
    heapify(T, wn, 0);
  }
}
/****************Tri Rapide****************/
  void permuter(array t, long a, long b) {
  long temp;
  temp = t[a];
  t[a] = t[b];
  t[b] = temp;
}
//pivot au millieu
long partition_1(array tab, long deb, long fin) {
  long p = deb;
  long pivot = tab[deb];
  for (long i = deb + 1; i <=fin; i++) {
    if (tab[i] < pivot) {
      p++;
      permuter(tab, i, p);
    }
  }
  permuter(tab, p, deb);
  return (p);
}

void tri_rapide_1(array tab, long deb, long fin) {
  if (deb < fin) {
    long piv = partition_1(tab, deb, fin);
    tri_rapide_1(tab, deb, piv - 1);
    tri_rapide_1(tab, piv + 1, fin);
  }
}
//pivot a la fin
long partition_2(array tab, long deb, long fin) {
  long p = fin;
  long pivot = tab[fin];
  for (long i = fin - 1; i >=deb; i--) {
    if (tab[i] > pivot) {
      p--;
      permuter(tab, i, p); 
    }
  }
  permuter(tab, p, fin);
  return (p);
}

void tri_rapide_2(array tab, long deb, long fin) {
  if (deb < fin) {
    long piv = partition_2(tab, deb, fin);
    tri_rapide_2(tab, deb, piv - 1);
    tri_rapide_2(tab, piv + 1, fin);
  }
}
  

//pivot au millieu
void tri_rapide_3(long list[], long left, long right)
{
    if (left >= right) return;
    long mid = (left + right) / 2;
    long i = left, j = right;
    long pivot = list[mid];
    long k = 0;

    while (left <= right) {
        while (list[left] < pivot) left++;
        while (list[right] > pivot) right--;
        if (left <= right) {
            permuter(list,left,right);
            left++; right--;
        }
    }
    tri_rapide_3(list, i, right);
    tri_rapide_3(list, left, j);
}
int main(void) {
  
  long Ns[8] = {10000,   50000,   100000,   500000,
                1000000, 5000000};
  FILE *f, *fptr;
  clock_t t1, t2;
  double delta;
  fptr = fopen("log rand bullr.txt", "w");
  for (int i = 0; i < 4; i++) {
    f = fopen("tablerand.txt", "r");
    for (long j = 0; j < Ns[i]; j++) {
      //T[j] = j;
      fscanf(f, "%li", &T[j]);
    }
    t1 = clock();
    printf("%li com\n",bubsort(T, Ns[i]));
    t2 = clock();
    delta = (double)(t2 - t1) / CLOCKS_PER_SEC;
    fprintf(fptr, "%f ,", delta);
    printf("%li items %f s\n ",Ns[i],delta);
  }
  fclose(fptr);
  for (long i = 0; i < 100; i++) {
    printf("%li\n", T[i]);
  }
  return 0;
}


\end{minted}