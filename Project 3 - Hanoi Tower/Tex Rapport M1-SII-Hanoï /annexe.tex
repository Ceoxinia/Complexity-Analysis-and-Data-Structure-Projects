\chapter{Annexe :code source}
\section{Repartition des taches}
\par
\small
\resizebox{17cm}{!}{
\begin{tabular}{| c | c | }
    \hline
    Member &  Tache   \\
    \hline
    BEBNBACHIR Mohamed Amir & la conception de la solution iterative de hanoi + l'implementation en langage C + etude experimentale et analyse des resultats de la version iterative\\
    \hline
    BOUCHOUL Bouchra &  L'algorithme de verification avec calcule de sa complexite + Presentation d'une instance du probleme avec la solution (exemple)  \\
    \hline
    KHEMISSI Maroua &  Historique et presentation du probleme + Definition formelle du probleme+  Presentation de l'algorithme de resolution avec complexite theorique + representation de la modelisation de la solution\\
    \hline
    MEDJKOUNE Roumaissa & le programme de l'algorithme de résolution récursive en langage c + Son étude expérimentale + Analyse des résultats  \\
    \hline
  
\end{tabular}}
\\ \\
\newpage
\section{algorithme.c}
Le programme c 
\begin{minted}[
breaklines=true,
frame=lines,
linenos
]{c}
#include<stdio.h>
#include<stdlib.h>
#include<string.h>
#define MAX 100
#include<time.h>
//pile
struct discs {
	int size;
};
typedef discs *disc;
//les piles
typedef struct pile
{
    disc items[MAX];
    int top;
    int taille;
    int num;
} pile;

typedef struct Tour{
	int sizej;
	pile *A;
	pile *B;
	pile *C;
}Tour;
typedef Tour *TourH;


//les fonctions des piles
void initPile(pile *p,int num)
{
    p->top = -1;
    p->taille = 0;
    p->num=num;
}
int isempty(pile *p) {
   if (p->top == -1)
        return 1;
    else
        return 0;
}
int pilePleine(pile *s)
{
    if (s->top == MAX - 1)
        return 1;
    else
        return 0;
}
void push(pile *p, disc bt) {
 
    if (pilePleine(p))
    {
        printf("la pile est pleine");
    }
    else
    {
        p->top++;
        p->items[p->top] = bt;
    }
    p->taille++;

}
disc pop(pile *p) {
  if (isempty(p)){
  	return NULL;
  }
    p->taille--;
    disc val = p->items[p->top];
    p->top--;
    return val;
    }

///Initialiser le jeux en precisant le nombre initial des disques
TourH initHanoi(int size){
	TourH Tr= (TourH)malloc(sizeof(TourH));
	pile *A = (pile *)malloc(sizeof(pile));
    initPile(A,1);
    pile *B = (pile *)malloc(sizeof(pile));
    initPile(B,2);
    pile *C = (pile *)malloc(sizeof(pile));
    initPile(C,3);
    int i;
    for (i=size;i>0;i--){
    	disc d = (disc)malloc(sizeof(disc));
    	d->size=i;
    	push(A,d);
	}
    Tr->sizej=size;
    Tr->A=A;
    Tr->B=B;
    Tr->C=C;
 return Tr;
}

///fonction pou deplacer les disque entre les piles

void move(pile *p, pile *k,int* dep){
	disc d;
	d=pop(p);
	push(k,d);
	(*dep) += 1;
	
}

///fonctions tour d'hanoi
void Hanoi(int n,pile *A,pile *B,pile *C,int* dep){
	if (n>0){
	   Hanoi(n-1,A,C,B,dep);
	   move(A,C,dep);
	   Hanoi(n-1,B,A,C,dep);
	
	}
}

void moveI(pile * from ,pile * to){

    if (isempty(to))
    {
        push(to,pop(from));
    }
 
    else if (isempty(from))
    {
        push(from,pop(to));
    }
 
    else if (from->items[from->top]>to->items[to->top])
    {
        push(from,pop(to));
    }
 
    else
    {
        push(to,pop(from));
    }
}
void iterativeHanoi(int n,pile* start,pile* end,pile* inter){
    
    int total_num_of_moves = pow(2, n) - 1;
    
    
    if (n % 2 != 0)
    {
 
    for (int i = 1; i <= total_num_of_moves; i++)
    {
        if (i % 3 == 1)
        moveI(start,end);
 
        else if (i % 3 == 2)
        moveI(start,inter);
 
        else if (i % 3 == 0)
        moveI(inter,end);
    }
    }else{
        for (int i = 0; i < total_num_of_moves; i++)
    {
        if (i % 3 == 1)
        moveI(start,end);
 
        else if (i % 3 == 2)
        moveI(end,inter);
 
        else if (i % 3 == 0)
        moveI(inter,start);
    }
    }
}
void TourHanoi(TourH Tr,int* dep){
	Hanoi(Tr->sizej,Tr->A,Tr->B,Tr->C,dep);
}

int main(){
//enregistrement des executions

 clock_t t1,t2;
 double time;
 int j=5;
 FILE *f;
 f = fopen("nbDepHanoiRec.txt", "a");
  if (f != NULL)
    {  printf(" le fichier existe \n");}
    else
    { printf("Impossible d'ouvrir le fichier \n");}

 while(j<100){
     int d=0;
     TourH jeu= initHanoi(j);
     t1=clock();
     TourHanoi(jeu,&d);
     t2=clock();
     time=(float)(t2 - t1) / CLOCKS_PER_SEC;
     fprintf(f,"%d %d \n",j,d);
     j+=5;
 }


return 0;
}

\end{minted}