\chapter{Introduction}

Un nombre premier est un nombre sans diviseurs autres que 1 et lui-même, par exemple : 2, 3, 5, 7, 11, 13, 17, 19. 
\par
Savoir rapidement si un nombre entier est un nombre premier ou non est un problème posé depuis des millénaire. De très nombreux algorithmes avancés ont été conceptualisés certains sont d'une complexité polynomiale.
\par
Dans ce travail nous allons tester 6 différents algorithmes et mesurer la difficulté de chacun par étudier ses deux complexités théorique et expérimentale en évaluant le temps nécessaire pour l'effectuer sur un ordinateur en fonction de la taille des données.
\\
