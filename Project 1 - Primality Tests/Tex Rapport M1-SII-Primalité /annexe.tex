\chapter{Annexe}

\section{Repartition des taches}
\par
\small
\resizebox{17cm}{!}{
\begin{tabular}{| c | c | }
    \hline
    Member &  Tache   \\
    \hline
    BEBNBACHIR Mohamed Amir & Analyse du temps d’exécution des nombres premiers ayant au plus 12 chiffres  \\
    \hline
    BOUCHOUL Bouchra & Execution et mesure de temps d'execution , preparation des donnees  \\
    \hline
    KHEMISSI Maroua & Analyse du temps d’exécution des 20 nombres premiers ayant la meme taille  \\
    \hline
    MEDJKOUNE Roumaissa & Analyse du temps d'execution apres 30 tests   \\
    \hline
    en collaboration de tous les membres & implémenter les algorithmes et rédige le rapport    \\
    \hline
\end{tabular}}
\\ \\
\newpage
\section{main.c}
\begin{minted}[
breaklines=true,
frame=lines,
linenos
]{c}
#include <math.h>
#include <stdio.h>
#include <time.h>
//Sol1 On essaie les divisions par 2,3..n-1 et on s'arrête au premier diviseur
int A1(unsigned long long int n) {
  for (unsigned long long int i = 2; i <= n-1; i++) {
    if (n % i == 0) {
      return 0;
    }
  }
  return 1;
}
//Sol2 pour accélérer la recherche, on essaie les divisions par 2,3..n/2 et on s'arrête au premier diviseur ou n/2
int A2(unsigned long long int n) {
  for (unsigned long long int i = 2; i <= (n / 2); i++) {
    if (n % i == 0) {
      return 0;
    }
  }
  return 1;
}
//Sol3 pour accélérer encore la recherche on essaie les divisions par 2,3..√n et on s'arrête au premier diviseur ou √n
int A3(unsigned long long int n) {
  for (unsigned long long int i = 2; i <= sqrt(n); i++) {
    if (n % i == 0) {
      return 0;
    }
  }
  return 1;
}

//Sol4 On élimine les multiples de 2 et on essaie pas les nombres pairs ,on s'arrête au premier diviseur
int A4(unsigned long long int n) {
  if (n != 2 && n % 2 == 0) {
    return 0;
  }
  if (n != 2) {
    for (unsigned long long int i = 3; i <= n - 2; i += 2) {
      if (n % i == 0) {
        return 0;
      }
    }
  }
  return 1;
}
//Sol5 On élimine les multiples de 2 et on essaie pas les nombres pairs ,on s'arrête au premier diviseur ou n/2
int A5(unsigned long long int n) {
  if (n != 2 && n % 2 == 0) {
    return 0;
  }
  if (n != 2) {
    for (unsigned long long int i = 3; i <= n / 2; i += 2) {
      if (n % i == 0) {
        return 0;
      }
    }
  }
  return 1;
}
//Sol6 On élimine les multiples de 2 et on essaie pas les nombres pairs ,on s'arrête au premier diviseur ou √n
int A6(unsigned long long int n) {
  if (n != 2 && n % 2 == 0) {
    return 0;
  }
  if (n != 2) {
    for (unsigned long long int i = 3; i <= sqrt(n); i += 2) {
      if (n % i == 0) {
        return 0;
      }
    }
  }
  return 1;
}
int main(void) {
  unsigned long long int T[7] = {314159,      4480649,    50943779,
                                 999999937,   5915587277, 41996139943,
                                 100123456789};
  unsigned long long int T2[20] = {1500450271,
1879048183,
1979339339,
2030444287,
2860486313,
3267000013,
3367900313,
3628273133,
4093082899,
4392489041,
5038465463,
5463458053,
5555333227,
5754853343,
5915587277,
6668999101,
7896879971,
8278487999,
9471413089,
9576890767};
  unsigned long long int big = 12764787846358441471U;
  clock_t t1, t2;
  double delta, s;
  int r = 0;
  FILE *fptr, *f;
  f= fopen("data.txt","w");
  printf("A1\n");
  for (int j = 0; j <20; j++ ){
    s = 0;
    for (int k = 0; k < 1; k++) {
      t1 = clock();
      r += A6(T2[j]);
      t2 = clock();
      delta = (double)(t2 - t1) / CLOCKS_PER_SEC;
      s += delta;
    }
    fprintf(f, "%f\n", s);
    printf("x%i %f s\n", j + 1, s);
  }
  fclose(f);
  fptr = fopen("A5test.txt", "w");
  printf("A5test1\n");
  for (int i = 0; i < 7; i++) {
    s = 0;

    for (int k = 0; k < 1; k++) {
      t1 = clock();
      r += A5(T[i]);
      t2 = clock();
      delta = (double)(t2 - t1) / CLOCKS_PER_SEC;
      s += delta;
    }
    fprintf(fptr, "%i digit number %f s\n", i + 6, s);
    printf("%i digit number %f s\n", i + 6, s);
  }
  t1 = clock();
  r += A6(big);
  t2 = clock();
  delta = (double)(t2 - t1) / CLOCKS_PER_SEC;
  fprintf(fptr, "%i digit number %f s\n", 20, delta);
  printf("%i digit number %f s %i\n", 20, delta, r);
  fclose(fptr);
  return 0;
}
\end{minted}