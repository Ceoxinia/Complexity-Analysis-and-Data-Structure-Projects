\chapter{Conclusion}

Après une mise à l'échelle des graphes afin d'éviter de laisser transparaître les différences de puissances de nos machines respectives, nous constatons que il y a une relation de correlation directe entre le nombre de chiffres et la moyenne du temps d'éxecution (plus que le nombre de chiffres augemente plus que le temps d'éxecution augemente) qui est le cas pour les algorithmes A1,A2,A4,A5. par contre les algorithmes A6 , A3 proposent une complexite optimale pour les tests de primalites quelque soit la taille de nombre le temps d'execution est toujours minimal. 
