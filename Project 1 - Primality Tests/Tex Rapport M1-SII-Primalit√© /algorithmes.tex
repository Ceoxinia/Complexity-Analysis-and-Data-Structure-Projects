\chapter{Les algorithmes utilisées}
\section{Algorithme A1}
 Sachant qu’un nombre premier n est un nombre entier qui n’est divisible que par 1 et par lui-même. L’algorithme A1 va donc consister en une boucle dans laquelle on va tester si le nombre n est divisible par 2, 3, ..., n-1.
\par
Nous pouvons le représenter via le pseudo code suivant :
\par
\begin{function}[H]
    \textbf{Variables :}\\
    i : entier\;
    premier : bool\;
    \Begin{
        $premier \leftarrow vrai $\;
        $i \leftarrow 2 $\;

         \While{$premier \leftarrow vrai \:\: ET \:\: i \: \le \: n-1$}
       {
            
            \uIf{$n \:\: mod \:\: 2== 0$}{
                $premier \leftarrow false $\;
                
            }
            \Else{
            $i++$\;
            }
        }

}
    \caption{A1(Entrée:n: entier;)}
\end{function}
\newpage
\section{Algorithme A2}
 Sachant que si n est divisible par 2, il est aussi divisible par n/2 et s’il est divisible par 3, il est aussi divisible par n/3. De manière générale, si n est divisible par i pour i = 1 ... [n/2] où [n/2] dénote la partie entière de n/2, il est aussi divisible par n/i. 
 \par
Nous pouvons représenter une optimisation de A1 via le pseudo code suivant :

\par
\begin{function}[H]
    \textbf{Variables :}\\
    i : entier\;
    premier : bool\;
    \Begin{
        $premier \leftarrow vrai $\;
        $i \leftarrow 2 $\;

         \While{$premier \leftarrow vrai \:\: ET \:\: i \: \le \: n/2$}
       {
            
            \uIf{$n \:\: mod \:\: 2== 0$}{
                $premier \leftarrow false $\;
                
            }
            \Else{
            $i++$\;
            }
        }

}
    \caption{A2(Entrée:n: entier;)}
\end{function}
\newpage

\section{Algorithme A3}
 Si n est divisible par x, il est aussi divisible par n/x. Il serait intéressant d’améliorer A2 en ne répétant le test de la divisibilité que jusqu’à x = n/x.
 \par
Nous pouvons représenter cette amélioration de A2 via le pseudo code suivant :

\par
\begin{function}[H]
    \textbf{Variables :}\\
    i : entier\;
    premier : bool\;
    \Begin{
        $premier \leftarrow vrai $\;
        $i \leftarrow 2 $\;

         \While{$premier \leftarrow vrai \:\: ET \:\: i \: \le \: \sqrt{n}$}
       {
            
            \uIf{$n \:\: mod \:\: 2== 0$}{
                $premier \leftarrow false $\;
                
            }
            \Else{
            $i++$\;
            }
        }

}
    \caption{A3(Entrée:n: entier;)}
\end{function}
\newpage

\section{Algorithme A4}
 Dans le cas où n est impair, il ne faut tester la divisibilité de n que par les nombres impairs.
 \par
Nous pouvons le représenter via le pseudo code suivant :

\par
\begin{function}[H]
    \textbf{Variables :}\\
    i : entier\;
    premier : bool\;
    \Begin{
        $premier \leftarrow vrai $\;

         \If{$n <> 2 \:\: ET \:\: n \: mod \: 2 =0$}
       {
            $premier \leftarrow false $\;
        }    
         \Else {
         \If{$n <> 2$}
         {
            $i \leftarrow 3 $\;
            \While{$premier \leftarrow vrai \:\: ET \:\: i \: \le n-2 \: $}
       {
            
            \uIf{$n \:\: mod \:\: i== 0$}{
                $premier \leftarrow false $\;
                
            }
            \Else{
            $i \leftarrow i+2$\;
            }
        }
            }
        }
        

}
    \caption{A4(Entrée:n: entier;)}
\end{function}
\newpage
\section{Algorithme A5}
 Une hybridation des algorithmes A2 et A4.
 \par
Nous pouvons le représenter via le pseudo code suivant :

\par
\begin{function}[H]
    \textbf{Variables :}\\
    i : entier\;
    premier : bool\;
    \Begin{
        $premier \leftarrow vrai $\;

         \If{$n <> 2 \:\: ET \:\: n \: mod \: 2 =0$}
       {
            $premier \leftarrow false $\;
        }    
         \Else {
         \If{$n <> 2$}
         {
            $i \leftarrow 3 $\;
            \While{$premier \leftarrow vrai \:\: ET \:\: i \: \le n/2 \: $}
       {
            
            \uIf{$n \:\: mod \:\: i== 0$}{
                $premier \leftarrow false $\;
                
            }
            \Else{
            $i \leftarrow i+2$\;
            }
        }
            }
        }
        

}
    \caption{A5(Entrée:n: entier;)}
\end{function}
\newpage
\section{Algorithme A6}
 Une hybridation des algorithmes A3 et A4.
 \par
Nous pouvons le représenter via le pseudo code suivant :

\par
\begin{function}[H]
    \textbf{Variables :}\\
    i : entier\;
    premier : bool\;
    \Begin{
        $premier \leftarrow vrai $\;

         \If{$n <> 2 \:\: ET \:\: n \: mod \: 2 =0$}
       {
            $premier \leftarrow false $\;
        }    
         \Else {
         \If{$n <> 2$}
         {
            $i \leftarrow 3 $\;
            \While{$premier \leftarrow vrai \:\: ET \:\: i \: \le \sqrt{n} \: $}
       {
            
            \uIf{$n \:\: mod \:\: i== 0$}{
                $premier \leftarrow false $\;
                
            }
            \Else{
            $i \leftarrow i+2$\;
            }
        }
            }
        }
        

}
    \caption{A6(Entrée:n: entier;)}
\end{function}

\section{L'evaluation theorique de la complexite}
\resizebox{17cm}{!}{
\begin{tabular}{| c | c | c | c |}
    \hline
    Algorithme &  Nombre maximum d’itération ( en fonction de n) & Complexité théorique & Nombre réel d’itération avec n=69524913 \\
    \hline
    A1 & n-2 & O(n) & 69524913  \\
    \hline
    A2 & [n/2]-1 & O(n) & 34762455  \\
    \hline
    A3 & \sqrt{n}-1 & O(\sqrt{n}) & 8337  \\
    \hline
    A4 & [n/2]-1 & O(n) & 34762455 \\
    \hline
    A5 & [n/4]-1 & O(n) & 17381227 \\
    \hline
    A6 & [\sqrt{n}/2]-1 & O(\sqrt{n}) & 4168  \\
    \hline
\end{tabular}}
\\ \\ 
